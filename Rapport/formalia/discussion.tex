\chapter{Discussion} \label{chap:Discussion}
This chapter will contain a discussion of the project, discussing the ways errors are handled, the use of dot notation and the requirements that have and haven't been implemented, from the MoSCoW analysis in \ref{sec:MoSCoW}. \newline

\textbf{\Large{Error handling}}

In purpose of error handling DatLanguage throws errors whenever an error occurs. Errors will terminate the program and print out the assigned message for the specific error.
An other options for error handling would be exceptions, which would make it possible for multiple errors to be reported at the same time. Also the group could have written their own exception handler, which would give more control for error reporting. The ANTLR tool throws exceptions, so a combined exception handler would have been possible. \newline

\textbf{\Large{Dot notation}}

DatLanguage is an imperative language, but with an object oriented feeling because of the dot notation on Tank, Gun, Radar, Battlefield and Utils. The dot notation is intended to give familiarity to the dot notation in object oriented languages. This might lead to misinterpretations since DatLanguage doesn't implement classes as object oriented languages do.\newline 

\textbf{\Large{MoSCoW}}

In section \ref{sec:MoSCoW} the MoSCoW analysis was created and described. The method was used to prioritize the requirements for the project's language. Table \ref{moscowDis} shows which of the requirements that have been implemented and which have not been implemented. \newline
All \textit{Must have} requirements has been implemented as they should, since these were the most important and essential requirements for the language. These were needed to make the logic, movement etc., for the robot. \newline
All \textit{Should have} requirements has also been implemented. The Events and Void and type methods helps the users in order to make more intelligent robots, where Comments are useful for the readability of the program, made by the user. \newline
Most of the requirements in \textit{Could have}, has been implemented, which is Print statements, Strings and Setup block. Print statements and Strings have been implemented for the user to be able to debug their code. The Setup block is used for code, that should only be run once each round. Here the user could write code to make some initializing movements e.g. make the  robot move to the wall. Cos, Sin \& Tan and For loops were not implemented in DatLanguage. \newline
None of the requirements from \textit{Won't have} were implemented in this project.

\begin{table}[H]
\centering
\begin{tabular}{ |l|l|l| }
\hline
\multicolumn{3}{ |c| }{MoSCoW analysis} \\
\hline
& MoSCoW items & Status \\
\hline
\multirow{7}{*}{Must have} & Primitive types and variables & Implemented \\
& While loop & Implemented \\
& Reserved calls & Implemented \\
& Robot naming & Implemented \\
& If/Else/Elseif statements & Implemented \\
& Arithmetic expressions and operators & Implemented \\
& Logical expressions and operators & Implemented \\ \hline
\multirow{3}{*}{Should have} & Events & Implemented \\
& Comments & Implemented \\
& Void and type methods & Implemented \\ \hline
\multirow{6}{*}{Could have} & Cos, Sin \& Tan & {\color{red}Not Implemented} \\
& For loops & {\color{red}Not Implemented} \\
& Print statements & Implemented \\
& Strings & Implemented \\
& Setup block & Implemented  \\ \hline
\multirow{3}{*}{Won't have} & Random number generator & {\color{red}Not Implemented} \\
& Other robot types & {\color{red}Not Implemented} \\
\hline
\end{tabular}
\caption{Fulfilment of the MoSCoW analysis}
\label{moscowDis}

\end{table}
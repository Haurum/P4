\chapter*{Preface}
This project has been developed as part of the fourth semester project by project group SW408F16 from Aalborg University, Software Engineering, from the period 1st February to 26th May 2016 . \newline
The project is based on the \textit{Aalborg-model}, study method, where problem and project based learning is the focus. The theme of this semester was to create a compiler for a new language. To do so, some subjects were introduced and the subject the group chosen, was \textit{Domain Specific Language for Robocoders }. \newline

The group would like to thank supervisor, Giovanni Bacci for his very much appreciated advice and guidance during the whole project. 
\newline
\newline
\newline
\newline

{\Huge\textbf{Signatures}}
\newline
\newline

\begin{table}[H]
	\centering
		\begin{tabular}{c c c}
			\underline{\phantom{mmmmmmmmmmmmmm}} & \underline{\phantom{mmmmmmmmmmmmmm}} & \underline{\phantom{mmmmmmmmmmmmmm}} \\
			Christian Dannesboe			& Frederik Børsting Lund 		& Karrar Al-Sami 			\\
			&&\\
			&&\\
			\underline{\phantom{mmmmmmmmmmmmmm}} & \underline{\phantom{mmmmmmmmmmmmmm}} & \underline{\phantom{mmmmmmmmmmmmmm}} \\
			Mark Kloch Haurum			& Emil Bønnerup 		& Søren Lyng 				\\
			&&\\
			&&\\
		 																		
		\end{tabular}
\end{table}

\chapter*{Reading guide}
This project has followed the courses Syntax and Semantics \& Languages and Compiler. The context of this project has been written according to the order the course materials was taught and learned. \newline

The sources in the report are being referred to by the Harvard citation method. This includes a last name and a publication year in the report, and in the \textit{\textbf{Bibliography}} chapter all the used sources are listed in alphabetical order. \newline
\textit{An example of a source in the text could be: \textbf{\citep{Sebesta}}.}
\newline
If the source is on the left side of a period, then that source refers only to that sentence and if the source is on the right side of a period, then it refers to the whole section. 

Figures and tables are referred to as a number. The number is determined by the chapter and the number of figure it appears as. \newline
\textit{For example: The first figure in a chapter will have the number \textbf{x.1}, where x is the number of the chapter. The next figure, will have the number \textbf{x.2}, etc.}
\newline
The listings of source code are also referred to as the tables and figures. 

Source code in the report are listed as code snippets, and they're not necessarily the same as the source code, meaning that code snippets may be shorter than the actual source code or missing comments from the source code. In order to show that, the use of the following three periods are used: \textit\textbf{{"..."}}, which means that some of the source code isn't listed in the code snippet, as it may be long and irrelevant. 


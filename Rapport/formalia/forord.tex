\chapter*{Forord}

Denne rapport er udarbejdet af en gruppe studerende på 1. semester på Byggeri og Anlægs-uddannelsen ved Aalborg Universitet. \textit{Byggeboom i Aalborg} er det overordnede tema for projektet.

Fra projektkataloget er der valgt projektet \textit{Energirenovering}, som lægger op til at belyse andre sider af et byggeboom. Projektet omfatter en kontekstuel vinkel og en teknisk vinkel. Den tekniske del belyser faglighederne energi og indeklima samt konstruktion. Den konstekstuelle del af rapporten behandler ...

Forudsætningerne for at læse rapporten er et vist kendskab til ... \\
Der rettes stor tak til vejlederne ... for inspirerende vejledning og konstruktiv kritik. Endvidere rettes en stor tak til ...

\textbf{Læsevejledning}

Der vil igennem rapporten fremtræde kildehenvisninger, og disse vil være samlet i en kildeliste bagerst i rapporten. Der er i rapporten anvendt kildehenvisning efter Harvardmetoden, så i teksten refereres en kilde med [Efternavn, År]. Denne henvisning fører til kildelisten, hvor bøger er angivet med forfatter, titel, udgave og forlag, mens Internetsider er angivet med forfatter, titel og dato. Figurer og tabeller er nummereret i henhold til kapitel, dvs. den første figur i kapitel 7 har nummer 7.1, den anden, nummer 7.2 osv. Forklarende tekst til figurer og tabeller findes under de givne figurer og tabeller.

\phantom{Luft}

\phantom{Luft}

\begin{table}[H]
	\centering
		\begin{tabular}{c c c}
			\underline{\phantom{mmmmmmmmmmmmmm}} & \underline{\phantom{mmmmmmmmmmmmmm}} & \underline{\phantom{mmmmmmmmmmmmmm}} \\
			Adam  G. Hansen			& Berit Jørgensen 		& Christoffer Haning 			\\
			&&\\
			&&\\
			\underline{\phantom{mmmmmmmmmmmmmm}} & \underline{\phantom{mmmmmmmmmmmmmm}} & \underline{\phantom{mmmmmmmmmmmmmm}} \\
			Dorthe Møller			& Ejnar V. Jensen 		& Freja Poulsen 				\\
			&&\\
			&&\\
		 							& \underline{\phantom{mmmmmmmmmmmmmm}} 	&			\\														
									& Gerhard Pedersen 							& 												
		\end{tabular}
\end{table}
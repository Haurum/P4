\chapter{Introduction}

In Danish high schools all kinds of different languages are available for the students, and the programming languages has found their place as well. For beginners, the syntactic rules, type systems and the nature of the language can be hard to comprehend at first, which is why this report will focus on constructing a language for the high school students. The interest for computer games in the 21st century is bigger than ever \citep{Wankel}, and computer games is a fun and educating approach to learning programming languages. Robocode is a game where the player has to code their own robot, giving every player the opportunity to battle each other’s robots, making it a competition of coding the ‘best’ robot. This is mainly coded in Java, which is an object oriented programming language, and this nature of the language can be hard to understand without having programming experience. There is no popular domain-specific language for Robocode, and therefore the high school students, who is not experienced programmers, may not be likely to code a working robot. This is due to the fact that it requires one to know an object oriented language in advance. 

This is a problem, since Robocode could potentially be a great way of introducing these students for programming languages. If there was a domain-specific language, with more intuitive type systems and good writability, students could easily be introduced for a programming language, and afterwards expand their knowledge gradually on a general purpose programming language. In this report, Robocode will be studied, and the final product should be a domain-specific language for Robocode, compiled to Java.
	
Based on the above mentioned introduction, the project will try to answer the following problem statement:

\textbf{How can a domain-specific language for Robocode simplify the creation of a robot for high school students, with little or no experience to programming?} 
\begin{itemize}
	\item How can the Java type system be simplified?
	\item Which constructs are necessary for programming standard robots in Robocode?
	\item How can an easy to use interface be made for the users?
\end{itemize}
\chapter{Language design}
In this chapter the design decisions made during the process of creating the language will be described here. There are three criteria for the development of the language, \emph{readability, writability} and \emph{reliability}. The decisions made to accommodate these will be described in detail in the first section of the chapter. 
In the next section, the MoSCoW method and the application of it in the design process will be described. 
 
\section{Language criteria}
In this section the three main criteria for designing the language will be discussed with focus on the implementation of these in the language. The criteria are based on theory from the book \emph{Concepts of Programming Languages}\citep{Sebesta}.

\subsection{Readability}
Readability is referring to the ease of reading and understanding a programming language. The language in this report should be very simple, since the programming language is, as earlier mentioned, targeted for high school students with little or no programming experience. Therefore the only the necessary features for a beginner in both programming and RoboCode should be implemented. 

Since the language is targeted for beginners, one of the criterias for the language would be to make the syntax as simple as possible, but still have it generally look like Java. The language should also have a high level of orthogonality, which also will help make the language simpler. 

\subsection{Writability}
The general purpose for a DSL language is a language is to be able to make solutions for a specific problem, therefore the writability is important in this project, since the purpose of this project is to make a DSL language for RoboCode. As mentioned in the section above, the language should have a high level of orthogonality, which will also help on  the writability of the language. 
Reliability

\subsection{Reliability}
REMEMBER TO INSERT!

\section{MoSCoW analysis}
INTRO

\textbf{Must have}
\begin{itemize}
\item Primitive types and variables (assignment)
\item While loop
\item Reserved calls
\item Robot naming
\item If/Else/Elseif statements)
\item Arithmetic expressions and operators
\item Logical expressions and operators
\end{itemize}
\textbf{Should have}
\begin{itemize}
\item Events
\item Void and type methods
\item Cos, Sin \& Tan
\end{itemize}
\textbf{Could have}
\begin{itemize}
\item For loops
\item Arrays
\item Strings
\item Print statements
\item Comments
\item Setup block
\end{itemize}
\textbf{Want to have, but can’t right now}
\begin{itemize}
\item Random number generator
\item Other robot types
\item Other RoboCode gamemodes
\end{itemize}

 

\chapter{Semantics}
This chapter provides a formal description of the language semantics. The chapter uses techniques and structures from the book \textit{Transitions and Trees} \cite{Huttel}.
 \section{Syntactic categories}
 To simplify the presentation of the semantics of our language, syntactic categories have been used. The syntactic categories are based on the grammer found in section \ref{sec:Grammar}. A collection of metavariables are presented in the following paragraphs which will be used throughout the chapter to present the type system and the operational semantics.
 
 \begin{math}
 e \in \textbf{Expr} - Expressions \newline
 S \in \textbf{Stmt} - Statement\newline
 n \in \textbf{Num} - Numerals\newline
 b \in \textbf{Bool} - Boolean\ Literal\newline
 B \in \textbf{Block} - Block \newline %sure about this??
 tx \in \textbf{Txt} - Txt\ Literal\newline
 t \in \textbf{Type} - Num,\ Bool\ and\ Txt\ types\newline
 vid \in \textbf{var} - Variable name \newline
 fid \in \textbf{Func} - Function name \newline
 aid \in \textbf{Act} - Action name \newline
 D_a \in \textbf{ActDcl} - Action \newline
 D_f \in \textbf{FuncDcl} - Function \newline
 D_v \in \textbf{VarDcl} - global\ variable\ declaration\newline
 acall \in \textbf{acall} - Action call \newline
 fcall \in \textbf{fcall} - Function call \newline
 rcall \in \textbf{rcall} - Predefined Robocode calls \newline
 ecall \in \textbf{ecall} - Event call \newline
 \end{math}
 
 \section{Formation rules}
 Missing intro!
 
\begin{math}
	B ::= \{ \ S \ \}
	\newline
	Stmt ::= \ vid = e \ | \ D_v \ | \ if \ e_1 \ B_1 \ ( elseif \ e_2 \ B_2 )^* \ (else B)^* \ \newline | \ Repeat \ While(e_1) \ B \ | \ Repeat \ B \ While \ (e_2) \ | \ Return \ e
	\newline
	Exprs ::= \ Expr \ op \ Exprs \ | \ \epsilon
	\newline
	Expr ::= \ ( \ - \ | \ NOT \ )? \ e 
	\newline
	D_a ::= \ aid(S) \ B
	\newline
	D_f ::= \ fid(S) \ B
	\newline
	D_v ::= \ vid; \ | \ vid \ = \ literal
	\newline
	Call ::= \ acall \ | \ fcall \ | \ vcall \ | ecall	
\end{math}
 
 \section{Semantic Functions}
 The purpose of the semantics functions is to change the syntactic elements to a semantic element. The semantic functions used in our language will be described in this section. 
  
  \subsection{Numeral Literals}
  The numbers literal in our language will be Numerals(with the symbol n, from the syntactic categories), which will be converted to real numbers with the semantic function: 
  
  \begin{math}
  \mathcal{N}: \textbf{Num} \rightarrow \mathbb{R}
  \end{math}
  
  Using this function, numerals as 
  \begin{math}
    \mathcal{N}
  \end{math}[\underline{5}] and 
  \begin{math}
    \mathcal{N}
  \end{math}[\underline{5.36}] will be mapped to the corresponding values 5 and 5.36. 
  
  
  \subsection{Text Literals}
  A Text literal is a sequence of symbols and characters in UTF-8(Unicode Transformation Format 8-bit) except the delimiter ("). The sequence of symbols and characters must be within the delimiter, for example "Hello world!". A new language \textbf{txtL} is defined, which purpose is to remove the quotation("). 
  
  \begin{math}
  txl \in \textbf{txtL} = (") \newline
  tx \in \textbf{Txt} = (") \ (U)^* \ (")\ \newline
  U \in \ {UTF-8}
  \end{math}
  
  The langauge \textbf{txtL} translates the input of the Text Literal to Text, which is done with the function below. 
  
  \begin{math}
  	\tau(\textbf{Txt}) \ \rightarrow \ \textbf{txtL} \newline
  	As \ an \ example: \ \tau("Insert \ text \ here") \ \rightarrow \ Insert \ text \ here
  \end{math}
  
  \subsection{Boolean Literals}
  The Boolean literals depict whether the expression is evaluated to true or false. The symbol for true is \begin{math} \top \end{math}, and the symbol for false is \begin{math} \bot \end{math} from the set of values from \textbf{bool} = {True, False}.
  
  \begin{math}
  	\beta : Bool \rightarrow bool
  \end{math}
  
  The semantic function above can be used to evaluate the semantic value of a boolean, take \begin{math} x \ is \ evaluated \ by \ the \ function \ \beta,  \end{math} as an example:
  
  \[ \beta(x) =
    \begin{cases}
      \top       & \quad \text{if } x \text{ is true}\\
      \bot  & \quad \text{if } x \text{ is false}\\
    \end{cases}
  \]
  
  
  \section{Environment Storemodel}
  MISSING INTRO!
  
  
  
  \section{Type system}
  THIS SHALL BE MOVED!!!
	\[
	[NUM] \quad
	\dfrac{}{\Gamma \vdash n ::= Num}
	\]
	 
	\[
	[BOOL] \quad
	\dfrac{}{\Gamma \vdash b ::= Bool}
	\]
	
  	\[
  	[TXT] \quad
  	\dfrac{}{\Gamma \vdash tx ::= Txt}
  	\]
  	
  \section{Operational semantic}
  
  \subsection{Declarations}
  	\[
	[EMPTY-VARDCL] \quad
	\dfrac{<\epsilon, \ E_v, \ st> \ \rightarrow_{Dv} \ <E_v, \ st> }{<var \ x; D_v,\ E_v,\ st>}
	\]
    	
   	\[
   	[VARDCL] \quad
   	\dfrac{<D_v, \ E_v^{''}, \ st[l \mapsto v]> \ \rightarrow_{Dv} \ <E_v^{'}, \ st^{'}> }{<var \ x = a; D_v,\ E_v,\ st> \rightarrow_{Dv} \ <E_v^{'}, \ st^{'}>}
   	\]
   	\begin{math}
 	  	\qquad \ Where: E_v, \ st \vdash a \rightarrow_A \ v
 	  	\qquad \ \ l = E_v(next)
 	  	\qquad \ \ E_v^{''} = E_v[x \mapsto l][next \mapsto new(l)]
   	\end{math}
    	
   	\[
   	[EMPTY-ACTDCL] \quad
   	\dfrac{E_v \vdash <\epsilon, \ E_a, \ st> \ \rightarrow_{Da} \ E_a, \ st}{ E_v, \vdash  <Action \ a \ is \ S; \ D_a, \ st, \ E_a> \ \rightarrow_{Da} E_a, \ st }
   	\]
    	
    	
  	\[
   	[ACTDCL] \quad
   	\dfrac{E_v \vdash <D_a, \ st[a \mapsto \ <S, \ E_v, \ E_a>]>   \ \rightarrow_{Da} st^{'}, E_a^{'} \ }{E_v \ \vdash \ <Action \ a \ is \ S; \ D_a, \ st, \ E_a> \ \rightarrow_{Da} \ st^{'}, \ E_a^{'}}
   	\]
    	
   	\[
   	[EMPTY-FUNCDCL] \quad
   	\dfrac{E_v \vdash < \epsilon, \ E_f, \ st> \ \rightarrow_{Df} \ E_f, \ st}{Function \ f \ is \ S; \ D_f, \ st \ \rightarrow_{Df} \ E_f, \ st}
   	\]
   	
   	\[
   	[FUNCDCL] \quad
   	\dfrac{E_v \vdash <D_f, \ st[f \mapsto <S, E_v, E_f>]> \rightarrow_{Df} st^{'}, \ E_f^{'}}{E_v \vdash <Function \ f \ is \ S; D_f, \ st, \ E_f > \rightarrow_{Df} st^{'}, \ E_f^{'}}
   	\]
   	
   	\subsection{Statements}
   	
   	\subsection{Expressions}
   	
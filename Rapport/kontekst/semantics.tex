\chapter{Semantics}
This chapter provides a formal description of the language semantics. The chapter uses techniques and structures from the book \textit{Transitions and Trees} \cite{Huttel}.
 \section{Syntactic categories}
 To simplify the presentation of the semantics of our language, syntactic categories have been used. The syntactic categories are based on the grammer found in section \ref{sec:Grammar}. A collection of metavariables are presented in the following paragraphs which will be used throughout the chapter to present the type system and the operational semantics.
 
 \begin{math}
 e \in \textbf{Expr} - Expressions \newline
 S \in \textbf{Stmt} - Statement\newline
 n \in \textbf{Num} - Numerals\newline
 b \in \textbf{Bool} - Boolean\ Literal\newline
 B \in \textbf{Block} - Block \newline %sure about this??
 tx \in \textbf{Txt} - Txt\ Literal\newline
 t \in \textbf{Type} - Num,\ Bool\ and\ Txt\ types\newline
 x \in \textbf{var} - Variable name \newline
 f \in \textbf{Func} - Function name \newline
 a \in \textbf{Act} - Action name \newline
 D_a \in \textbf{ActDcl} - Action \newline
 D_f \in \textbf{FuncDcl} - Function \newline
 D_v \in \textbf{VarDcl} - global\ variable\ declaration\newline
 \end{math}
 
 \section{Formation rules}
 Missing intro!
 
\begin{math}
	B ::= \{ \ S \ \}
	\newline
	Stmt ::= \ vid = e \ | \ D_v \ | \ if \ e_1 \ B_1 \ ( elseif \ e_2 \ B_2 )^* \ (else B)^* \ \newline | \ Repeat \ While(e_1) \ B \ | \ Repeat \ B \ While \ (e_2) \ | \ Return \ e
	\newline
	Exprs ::= \ Expr \ op \ Exprs \ | \ \epsilon
	\newline
	Expr ::= \ ( \ - \ | \ NOT \ )? \ e 
	\newline
	D_a ::= \ aid(S) \ B
	\newline
	D_f ::= \ fid(S) \ B
	\newline
	D_v ::= \ vid; \ | \ vid \ = \ literal
	\newline
	Call ::= \ acall \ | \ fcall \ | \ vcall \ | ecall	
\end{math}
 
 \section{Semantic Functions}
 The purpose of the semantics functions is to change the syntactic elements to a semantic element. The semantic functions used in our language will be described in this section. 
  
  \subsection{Numeral Literals}
  The numbers literal in our language will be Numerals(with the symbol n, from the syntactic categories), which will be converted to real numbers with the semantic function: 
  
  \begin{math}
  \mathcal{N}: \textbf{Num} \rightarrow \mathbb{R}
  \end{math}
  
  Using this function, numerals as 
  \begin{math}
    \mathcal{N}
  \end{math}[\underline{5}] and 
  \begin{math}
    \mathcal{N}
  \end{math}[\underline{5.36}] will be mapped to the corresponding values 5 and 5.36. 
  
  
  \subsection{Text Literals}
  A Text literal is a sequence of symbols and characters in UTF-8(Unicode Transformation Format 8-bit) except the delimiter ("). The sequence of symbols and characters must be within the delimiter, for example "Hello world!". 
  
  %\begin{math}
  %txl \in \textbf{txtL} = (") \newline
  %tx \in \textbf{Txt} = (") \ (U)^* \ (")\ \newline
  %U \in \ {UTF-8}
  %\end{math}
  
  Text Literal are interpreted as string elements to Text, which is done with the function below. 
  
  \begin{math}
  	\tau(\textbf{Txt}) \ \rightarrow \ \textbf{txtL} \newline
  	As \ an \ example: \ \tau("Insert \ text \ here") \ \rightarrow \ Insert \ text \ here
  \end{math}
  
  \subsection{Boolean Literals}
  The Boolean literals depict whether the expression is evaluated to true or false. The symbol for true is \begin{math} \top \end{math}, and the symbol for false is \begin{math} \bot \end{math} from the set of values from \textbf{bool} = {True, False}.
  
  \begin{math}
  	\beta : Bool \rightarrow bool
  \end{math}
  
  The semantic function above can be used to evaluate the semantic value of a boolean, take \begin{math} x \ is \ evaluated \ by \ the \ function \ \beta,  \end{math} as an example:
  
  \[ \beta(x) =
    \begin{cases}
      \top       & \quad \text{if } x \text{ is true}\\
      \bot  & \quad \text{if } x \text{ is false}\\
    \end{cases}
  \]
  
  
  \section{Environment Storemodel}
  MISSING INTRO!
  
  
  
  \section{Type system}
  THIS SHALL BE MOVED!!!
	\[
	[NUM] \quad
	\dfrac{}{\Gamma \vdash n ::= Num}
	\]
	 
	\[
	[BOOL] \quad
	\dfrac{}{\Gamma \vdash b ::= Bool}
	\]
	
  	\[
  	[TXT] \quad
  	\dfrac{}{\Gamma \vdash tx ::= Txt}
  	\]
  	
  \section{Operational semantic}
  
  \subsection{Declarations}
  	\[
	[EMPTY-VARDCL] \quad
	\dfrac{<\epsilon, \ E_v, \ st> \ \rightarrow_{Dv} \ <E_v, \ st> }{<var \ x; D_v,\ E_v,\ st>}
	\]
    	
   	\[
   	[VARDCL] \quad
   	\dfrac{<D_v, \ E_v^{''}, \ st[l \mapsto v]> \ \rightarrow_{Dv} \ <E_v^{'}, \ st^{'}> }{<var \ x = a; D_v,\ E_v,\ st> \rightarrow_{Dv} \ <E_v^{'}, \ st^{'}>}
   	\]
   	\begin{math}
 	  	\qquad \ Where: E_v, \ st \vdash a \rightarrow_A \ v
 	  	\qquad \ \ l = E_v(next)
 	  	\qquad \ \ E_v^{''} = E_v[x \mapsto l][next \mapsto new(l)]
   	\end{math}
    	
   	\[
   	[EMPTY-ACTDCL] \quad
   	\dfrac{E_v \vdash <\epsilon, \ E_a, \ st> \ \rightarrow_{Da} \ E_a, \ st}{ E_v, \vdash  <Action \ a \ is \ S; \ D_a, \ st, \ E_a> \ \rightarrow_{Da} E_a, \ st }
   	\]
    	
    	
  	\[
   	[ACTDCL] \quad
   	\dfrac{E_v \vdash <D_a, \ st[a \mapsto \ <S, \ E_v, \ E_a>]>   \ \rightarrow_{Da} st^{'}, E_a^{'} \ }{E_v \ \vdash \ <Action \ a \ is \ S; \ D_a, \ st, \ E_a> \ \rightarrow_{Da} \ st^{'}, \ E_a^{'}}
   	\]
    	
   	\[
   	[EMPTY-FUNCDCL] \quad
   	\dfrac{E_v \vdash < \epsilon, \ E_f, \ st> \ \rightarrow_{Df} \ E_f, \ st}{Function \ f \ is \ S; \ D_f, \ st \ \rightarrow_{Df} \ E_f, \ st}
   	\]
   	
   	\[
   	[FUNCDCL] \quad
   	\dfrac{E_v \vdash <D_f, \ st[f \mapsto <S, E_v, E_f>]> \rightarrow_{Df} st^{'}, \ E_f^{'}}{E_v \vdash <Function \ f \ is \ S; D_f, \ st, \ E_f > \rightarrow_{Df} st^{'}, \ E_f^{'}}
   	\]
   	
   	\subsection{Statements}
   	
   	\subsection{Expressions}
   	
   	\[
   	[PAR-EXPR] \quad
   	\dfrac{E_v, \ E_a, \ E_f \vdash <e, \ st> \rightarrow_e (v, \ st^{'}}{E_v, \ E_a, \ E_f \vdash <(e), \ st> \rightarrow_e \ (v, \ st^{'})}
   	\]
   	
   	\subsection{Arithmetic Expressions}
   	
   	\[
   	[ADD-EXPR] \quad
   	\dfrac{E_v, \ E_a, \ E_f \vdash <a_1, \ st> \rightarrow_{e} (v_1, st^{''}) \qquad E_v, \ E_a, \ E_f \vdash <a_2, \ st^{''}> \rightarrow_e (v_2, \ st^{'})}{E_v, \ E_a, \ E_f \vdash <a_1 + a_2, st> \rightarrow_{e} (v, st^{'})}
   	\]
 	\begin{math}
   	v \ = \ v_1 \ + \ v_2
   	\end{math}
   	
   	\[
   	[SUB-EXPR] \quad
   	\dfrac{E_v, \ E_a, \ E_f \vdash <a_1, \ st> \rightarrow_{e} (v_1, st^{''}) \qquad E_v, \ E_a, \ E_f \vdash <a_2, \ st^{''}> \rightarrow_e (v_2, \ st^{'})}{E_v, \ E_a, \ E_f \vdash <a_1 - a_2, st> \rightarrow_{e} (v, st^{'})}
   	\]
   	\begin{math}
   	v \ = \ v_1 \ - \ v_2
   	\end{math}
   	
   	\[
   	[MUL-EXPR] \quad
   	\dfrac{E_v, \ E_a, \ E_f \vdash <a_1, \ st> \rightarrow_{e} (v_1, st^{''}) \qquad E_v, \ E_a, \ E_f \vdash <a_2, \ st^{''}> \rightarrow_e (v_2, \ st^{'})}{E_v, \ E_a, \ E_f \vdash <a_1 * a_2, st> \rightarrow_{e} (v, st^{'})}
   	\]
   	\begin{math}
   	v \ = \ v_1 \ * \ v_2
   	\end{math}
   	
   	\[
   	[DIV-EXPR] \quad
   	\dfrac{E_v, \ E_a, \ E_f \vdash <a_1, \ st> \rightarrow_{e} (v_1, st^{''}) \qquad E_v, \ E_a, \ E_f \vdash <a_2, \ st^{''}> \rightarrow_e (v_2, \ st^{'})}{E_v, \ E_a, \ E_f \vdash <a_1 / a_2, st> \rightarrow_{e} (v, st^{'})}
   	\]
   	\begin{math}
   	v \ = \ v_1 \ / \ v_2
   	\end{math}   	   	

   	\[
   	[MOD-EXPR] \quad
   	\dfrac{E_v, \ E_a, \ E_f \vdash <a_1, \ st> \rightarrow_{e} (v_1, st^{''}) \qquad E_v, \ E_a, \ E_f \vdash <a_2, \ st^{''}> \rightarrow_e (v_2, \ st^{'})}{E_v, \ E_a, \ E_f \vdash <a_1 \% a_2, st> \rightarrow_{e} (v, st^{'})}
   	\]
   	\begin{math}
   	v \ = \ v_1 \ \% \ v_2
   	\end{math}   	

   	\subsection{Boolean Expressions}
   	\[
   	[EXPR^{IS=}_{\ \top}] \quad
   	\dfrac{E_v, \ E_a, \ E_f \vdash <e_1, \ st> \rightarrow_e (v_1, st^{''}) \qquad E_v, \ E_a, \ E_f \vdash <e_2, \ st^{''} \rightarrow_e (v_2, \ st^{'})}{E_v, \ E_a, \ E_f \ \vdash \ <e_1
   	\ IS= e_2, \ st> \ \rightarrow_e (\top, \ st^{'})}
   	\]
   	\begin{math}
   	v_1 \ = \ v_2 
   	\end{math}

   	\[
   	[EXPR^{IS=}_{\ \bot}] \quad
   	\dfrac{E_v, \ E_a, \ E_f \vdash <e_1, \ st> \rightarrow_e (v_1, st^{''}) \qquad E_v, \ E_a, \ E_f \vdash <e_2, \ st^{''} \rightarrow_e (v_2, \ st^{'})}{E_v, \ E_a, \ E_f \ \vdash \ <e_1
   	\ IS= e_2, \ st> \ \rightarrow_e (\bot, \ st^{'})}
   	\]
	\begin{math}
   	v_1 \ \not= \ v_2
   	\end{math}


   	\[
   	[EXPR^{NOT=}_{\ \top}] \quad
   	\dfrac{E_v, \ E_a, \ E_f \vdash <e_1, \ st> \rightarrow_e (v_1, st^{''}) \qquad E_v, \ E_a, \ E_f \vdash <e_2, \ st^{''} \rightarrow_e (v_2, \ st^{'})}{E_v, \ E_a, \ E_f \ \vdash \ <e_1 \ NOT= e_2, \ st> \ \rightarrow_e (\top, \ st^{'})}
   	\]
	\begin{math}
   	v_1 \ \not= \ v_2
   	\end{math}   	
   	

   	\[
   	[EXPR^{NOT=}_{\ \bot}] \quad
   	\dfrac{E_v, \ E_a, \ E_f \vdash <e_1, \ st> \rightarrow_e (v_1, st^{''}) \qquad E_v, \ E_a, \ E_f \vdash <e_2, \ st^{''} \rightarrow_e (v_2, \ st^{'})}{E_v, \ E_a, \ E_f \ \vdash \ <e_1 \ NOT= e_2, \ st> \ \rightarrow_e (\bot, \ st^{'})}
   	\]
	\begin{math}
   	v_1 \ = \ v_2
   	\end{math}   	
   	
   	
   	\[
   	[EXPR^{\ >}_{\ \top}] \quad
   	\dfrac{E_v, \ E_a, \ E_f \vdash <e_1, \ st> \rightarrow_e (v_1, st^{''}) \qquad E_v, \ E_a, \ E_f \vdash <e_2, \ st^{''} \rightarrow_e (v_2, \ st^{'})}{E_v, \ E_a, \ E_f \ \vdash \ <e_1 \ > e_2, \ st> \ \rightarrow_e (\top, \ st^{'})}
   	\]
	\begin{math}
   	v_1 \ > \ v_2
   	\end{math}   	
   	
   	\newcommand{\exprtrans}[5][E_v, E_a,E_f]{#1\vdash \langle #2, #3 \rangle \mathrel{\to_e} (#4 , #5)}
   	
   	\[
   	[EXPR^{\ >}_{\ \bot}] \quad
   	\dfrac{E_v, \ E_a, \ E_f \vdash <e_1, \ st> \rightarrow_e (v_1, st^{''}) \qquad E_v, \ E_a, \ E_f \vdash <e_2, \ st^{''} \rightarrow_e (v_2, \ st^{'})}{E_v, \ E_a, \ E_f \ \vdash \ <e_1 \ > e_2, \ st> \ \rightarrow_e (\bot, \ st^{'})}
   	\]
	\begin{math}
   	v_1 \ \not> \ v_2
   	\end{math}   
   	
   	\[
   	[EXPR^{\ <}_{\ \top}] \quad
   	\dfrac{E_v, \ E_a, \ E_f \vdash <e_1, \ st> \rightarrow_e (v_1, st^{''}) \qquad E_v, \ E_a, \ E_f \vdash <e_2, \ st^{''} \rightarrow_e (v_2, \ st^{'})}{E_v, \ E_a, \ E_f \ \vdash \ <e_1 \ < e_2, \ st> \ \rightarrow_e (\top, \ st^{'})}
   	\]
	\begin{math}
   	v_1 \ < \ v_2
   	\end{math}   	
   	

   	\[
   	[EXPR^{\ <}_{\ \bot}] \quad
   	\dfrac{E_v, \ E_a, \ E_f \vdash <e_1, \ st> \rightarrow_e (v_1, st^{''}) \qquad E_v, \ E_a, \ E_f \vdash <e_2, \ st^{''} \rightarrow_e (v_2, \ st^{'})}{E_v, \ E_a, \ E_f \ \vdash \ <e_1 \ < e_2, \ st> \ \rightarrow_e (\bot, \ st^{'})}
   	\]
	\begin{math}
   	v_1 \ \not< \ v_2
   	\end{math}
   	
   	\[
   	[EXPR^{\ >=}_{\ \ \top}] \quad
   	\dfrac{E_v, \ E_a, \ E_f \vdash <e_1, \ st> \rightarrow_e (v_1, st^{''}) \qquad E_v, \ E_a, \ E_f \vdash <e_2, \ st^{''} \rightarrow_e (v_2, \ st^{'})}{E_v, \ E_a, \ E_f \ \vdash \ <e_1 \ >= e_2, \ st> \ \rightarrow_e (\top, \ st^{'})}
   	\]
	\begin{math}
   	v_1 \ >= \ v_2
   	\end{math}   	
   	

   	\[
   	[EXPR^{\ >=}_{\ \ \bot}] \quad
   	\dfrac{E_v, \ E_a, \ E_f \vdash <e_1, \ st> \rightarrow_e (v_1, st^{''}) \qquad E_v, \ E_a, \ E_f \vdash <e_2, \ st^{''} \rightarrow_e (v_2, \ st^{'})}{E_v, \ E_a, \ E_f \ \vdash \ <e_1 \ >= e_2, \ st> \ \rightarrow_e (\bot, \ st^{'})}
   	\]
	\begin{math}
   	v_1 \ \not>= \ v_2
   	\end{math}
   	
   	\[
   	[EXPR^{\ <=}_{\ \ \top}] \quad
   	\dfrac{E_v, \ E_a, \ E_f \vdash <e_1, \ st> \rightarrow_e (v_1, st^{''}) \qquad E_v, \ E_a, \ E_f \vdash <e_2, \ st^{''} \rightarrow_e (v_2, \ st^{'})}{E_v, \ E_a, \ E_f \ \vdash \ <e_1 \ <= e_2, \ st> \ \rightarrow_e (\top, \ st^{'})}
   	\]
	\begin{math}
   	v_1 \ <= \ v_2
   	\end{math}   	
   	

   	\[
   	[EXPR^{\ <=}_{\ \ \bot}] \quad
   	\dfrac{E_v, \ E_a, \ E_f \vdash <e_1, \ st> \rightarrow_e (v_1, st^{''}) \qquad E_v, \ E_a, \ E_f \vdash <e_2, \ st^{''} \rightarrow_e (v_2, \ st^{'})}{E_v, \ E_a, \ E_f \ \vdash \ <e_1 \ <= e_2, \ st> \ \rightarrow_e (\bot, \ st^{'})}
   	\]
	\begin{math}
   	v_1 \ \not<= \ v_2
   	\end{math}
   	
   	\[
   	[EXPR^{\ \ !}_{\ \ \top}] \quad
   	\dfrac{E_v, \ E_a, \ E_f \vdash <v, \ st> \rightarrow_e (\top, st^{'})}{E_v, \ E_a, \ E_f \ \vdash \ <!v, \ st> \ \rightarrow_e (\top, \ st^{'})}
   	\]  	  	

   	\[
   	[EXPR^{\ \ !}_{\ \ \bot}] \quad
   	\dfrac{E_v, \ E_a, \ E_f \vdash <v, \ st> \rightarrow_e (\bot, st^{'})}{E_v, \ E_a, \ E_f \ \vdash \ <!v, \ st> \ \rightarrow_e (\bot, \ st^{'})}
   	\]
   	
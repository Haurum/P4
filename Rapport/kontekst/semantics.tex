\chapter{Semantics}
This chapter provides a formal description of the language semantics. The chapter uses techniques from the book \textit{Transitions and Trees} \citep{Huttel}.
	
\section{Syntactic categories}	
To simplify the presentation of the semantics of our language, syntactic categories have been used. The syntactic categories are based on the grammar found in section \ref{sec:Grammar}. A collection of meta variables are presented in the following paragraphs which will be used throughout the chapter to present the operational semantics.

\begin{math}		
		e \in \textbf{Expr} - Expressions \newline		
		S \in \textbf{Stmt} - Statements\newline	
		n \in \textbf{Num} - Numerals\newline	
		b \in \textbf{Bool} - Boolean\ literals\newline		
		B \in \textbf{Block} - Blocks \newline %sure about this??	
		tx \in \textbf{String} - String\ literals\newline	
		t \in \textbf{Type} - Num,\ Bool\ and\ String\ types\newline	
		x \in \textbf{var} - Variable \ name \newline	
		f \in \textbf{Func} - Function \ name \newline	
		a \in \textbf{Act} - Action \ name \newline	
		D_a \in \textbf{ActDcl} - Action\ declaration\newline
		D_f \in \textbf{FuncDcl} - Function\ declaration\newline
		D_v \in \textbf{VarDcl} - Gobal\ variable\ declaration\newline
		F \in \textbf{Battle-Field} - Battle-field\ configuration\newline
\end{math}
	
\section{Formation rules}
These rules have the purpose of defining how to format code in the language. It doesn't define how things works, this will be described later in the semantics.
	
\begin{math}	
	B ::= \{ \ D_v \ S \ \}	
	\newline	
	S ::= \ Stmt \ S \ | \ \epsilon
	\newline
	Stmt ::= \ x = e \ | \ if \ e_1 \ B_1 \ else \ B_2 \ \newline | \ Repeat \ While(e) \ B \ | \ Return \ e \ | \ skip \ | \ atom \ S \ end \ | active \ S \ end
	\newline
	a ::= \ a(Dv) \ B \ D_a \ | \epsilon 	
	\newline	
	D_f ::= \ f(Dv) \ B \ D_f \ | \epsilon 
	\newline
	D_v ::= \ \tau \ x \ D_v \ | \epsilon 
	\newline
	Game ::= \ Stmt \ || \ F
\end{math}
	
\section{Semantic functions}
The purpose of the semantics functions to interpret syntactic elements to semantic elements. The semantic functions used in our language will be described in this section. 
	
\subsection{Numeral literals}	
The numbers literal in our language will be Numerals, which will be interpreted to real numbers by means of the semantic function: 
		
\begin{math}
	\mathcal{N}: \textbf{Num} \rightarrow \mathbb{R}
\end{math}
	
Using this function, numerals as 
\begin{math}
	\mathcal{N}
\end{math}[\underline{5}] and 
\begin{math}
	\mathcal{N}		
\end{math}[\underline{5.36}] will be mapped to the corresponding values 5 and 5.36. 
	
\subsection{String literals}
A String literal is a sequence of symbols and characters in UTF-8(Unicode Transformation Format 8-bit) except the delimiter ("). The sequence of symbols and characters must be within the delimiter, for example "Hello world!". \newline
String literal are interpreted as string elements Text, by means of the following:
\begin{math}		
	\tau : \textbf{String} \ \rightarrow \ \textbf{StringL} \newline	
	As \ an \ example: \ \tau("Insert \ text \ here") \ \rightarrow \ Insert \ text \ here		
\end{math}
	
\subsection{Boolean literals}
The Boolean literals depict whether the expression is evaluated to True or False. The semantic value for true is \begin{math} \top \end{math}, and false is \begin{math} \bot. \end{math}
	
\section{Environment store model}
The environment store model describes how variables are bound to a storage during the execution of programs. The variables are bound to a storage-call, where the storage-cell contains the value of the variable. Storage-cells are also called \textit{locations}, the set of locations is denoted \textbf{Loc}.
	
The variable environment is a function that will tell what storage location a variable is bound to. Variable environments is the set of partial functions from the variable to their location, as shown in the following equation:

\begin{math} 		
	E_v \in \boldsymbol{EnvV} = \boldsymbol{Var} \cup \{ next \} \rightharpoonup \ \boldsymbol{Loc}		
\end{math}

We assume the \textit{next} element does not belong to \textit{var} and will be used to point to the next available location in memory.
	
Let l be an arbitrary element of \textbf{Loc} and assume that \textbf{Loc} = \begin{math} \mathbb{N}. \end{math} The function \textbf{new} found below, will return the successor for every location by defining \textbf{new} as: \textbf{new} l = l + 1.
	
\begin{math}		
	\boldsymbol{new} : \ \boldsymbol{Loc} \ \rightarrow \ \boldsymbol{Loc}		
\end{math}
	
Let \begin{math}E_v\end{math} be an arbitrary element in \textbf{EnvV}, we denote by \begin{math} E_v[x \mapsto l] \end{math} the variable environment \begin{math}E_v^{'} \end{math} defined as follows:
	\[ E^{'}_v(y) =	
	\begin{cases}
	E_v(y) & \quad \text{if } y \not= x\\
	l & \quad \text{if } y = x\\	
	\end{cases}	
	\]
\begin{math}Env_v[x \mapsto v] \end{math} represents the update of \begin{math} x \ \in \ var \ to \ l \ \in \ Loc.\end{math} 
	
A partial function in \textbf{St} is defined below with an arbitrary element st.
	
\begin{math}		
	st \ \in \boldsymbol{St} \ = \ \boldsymbol{Loc} \rightharpoonup \ \boldsymbol{Value}		
\end{math}
	
For updating the store st \begin{math} \in \end{math} \textbf{St} the update notation \begin{math} sto[l \mapsto v] \end{math} will be used. \begin{math} sto[l \mapsto v] \end{math} will denote the store function \begin{math}sto^{'} \end{math} defined as follows:
	\[ sto^{'}(l^{'}) =	
	\begin{cases}	
	sto(l) & \quad \text{if } l^{'} \not= l\\	
	v & \quad \text{if } l^{'} = l\\	
	\end{cases}	
	\]
	
\section{Operational semantic}
This section has the purpose of formally describing the language. It consists of semantic rules that give context to a program. \newline
The semantics is described with big-step semantics due to the intuitiveness of this kind of semantics. \newline
The environments during the operational semantics will be denoted with E.
	
\subsection{Blocks}
	\[	
	[BLOCK]	
	\dfrac{\dfrac{E_f, \ E_a \ \vdash \ \langle B, \ st^{''}, \ E_v^{''} \rangle \ \Rightarrow^* \ (st^{'}, \ E_l^{'})}{\dfrac{E_v, \ E_a, \ E_f \ \vdash \ \langle e_1, \ st_1 \rangle \ \rightarrow_e \ (v_1, \ st_2)}{E_v, \ E_a, \ E_f \ \vdash \ \langle e_n, \ st_n \rangle \ \rightarrow_e \ (v_n, \ st_{n+1})}}}{E_v, \ E_a, \ E_f \ \vdash \ \langle f(e_1 \ ... \ e_n), \ st \rangle \ \rightarrow_e \ (v, \ st^{'})}	
	\]
	
\begin{math}		
	\quad where \newline		
	\qquad st_1 = st \newline		
	\qquad E_f(f) = (B, x_1 ... x_n, E_v^{'}) \newline		
	\qquad E_v^{''} = E_v^{'}[x_1 \mapsto l_1] ... [x_n \mapsto l_n] [next \mapsto l_{n+1}] \newline		
	\qquad st^{''} = st_{n+1}[l_1 \mapsto v_1] ... [l_n \mapsto v_n] \newline		
	\qquad l_1 = new (Ev(next)) \newline		
	\qquad l_{i+1}= new l_i for i \in {1 ... n}		
\end{math}
	
\subsubsection{Declarations}	
	\[	
	[EMPTY-VARDCL] \quad	
	\dfrac{}{\langle\epsilon, \ E_v\rangle \rightarrow_{Dv} E_v}	
	\]\newline
	\[	
	[VARDCL] \quad	
	\dfrac{ \langle D_v, \ E_v[x \mapsto l][next \mapsto new_l]\rangle \ \rightarrow_{Dv} \ \langle E_v^{'} \rangle }{ \langle \ \tau \ x; \ D_v,\ E_v \rangle \rightarrow_{Dv} \ \langle E_v^{'} \rangle } \newline	
	\qquad \ \ l = E_v(next)	
	\]\newline
	\[	
	[EMPTY-ACTDCL] \quad	
	\dfrac{}{E_v \ \vdash \ \langle \epsilon, E_a \rangle \ \rightarrow_{Da} \ E_a}	
	\]\newline
	\[	
	[ACTDCL] \quad	
	\dfrac{E_v \ \vdash \langle D_a, \ E_a[a \ \mapsto (S, \ x_1 .. x_n \ E_v)] \rangle \rightarrow_{Da} \ E_a^{'}}{E_v \ \vdash \ Action \ a(x_1 .. x_n) \ is \ S; \ D_a \ E_a \ \rightarrow_{Da} \ E_a^{'}}	
	\]\newline
	\[	
	[EMPTY-FUNCDCL] \quad	
	\dfrac{}{E_v \ \vdash \langle \epsilon; \ D_f \rangle \ \rightarrow_{Df} \ E_f}	
	\]\newline
	\[	
	[FUNCDCL] \quad	
	\dfrac{E_v \vdash \langle D_f, \ E_f[f \ \mapsto (S, \ x_1 .. x_n, \ E_v)] \rangle \rightarrow_{Df} E_f^{'}}{E_v \vdash \ \langle Function \ f(x_1 .. x_n) \ is \ S; \ D_f, \ E_f \rangle \ \rightarrow_{Df} \ E_f^{'}}	
	\]

\subsection{Statements}
The interaction with the battlefield will not be explained in the semantics in this project, since it is out of the scope of the project. The interaction that a battlefield configuration may have with a statement running in parallel will be explained, but only for when it moves from one configuration to another.	
	
The configuration of the semantics statements are:
\begin{itemize}
	\item \begin{math} \langle \end{math}S, st, envl\begin{math} \rangle \end{math}, which are intermediate configurations where S is a statement that may contain atomic-encapsulations or parallels.
	\item \begin{math} \langle \end{math}st, envl\begin{math} \rangle \end{math} is terminal configurations.	
\end{itemize}
	
In intermediate configuration is where only parts of the statements will be interpreted. Where in terminal configurations all the statements will be executed.
	
The transition system of our language is:
\begin{math}		
	((\boldsymbol{S} \ \times \boldsymbol{EvCxt} \ \times \boldsymbol{St}) \ \cup \ (\boldsymbol{St} \ \times \ \boldsymbol{Envl}), \Rightarrow, \ \boldsymbol{St} \ \times \ \boldsymbol{Envl})		
\end{math}
	
The set of run time stacks \textbf{EvCxt} is:
\textbf{Envl} = \textbf{EnvV}\begin{math}		
		^*		
\end{math}
	
We will denote by \begin{math} E_l \end{math} a generic element in EvCxt
	
	\[	
	[ATOM-PARTLY]	
	\dfrac{E_l, \ E_f, \ E_a \ \vdash \langle S, \ st, \ E_l \rangle \ \Rightarrow \ \langle S^{''}, \ st^{'}, \ E_l^{'} \rangle}{E_e, \ E_f, \ E_a \ \vdash \ \langle atom \ S \ end, \ st, \ E_l \rangle \ \Rightarrow \ \langle atom \ S^{''} \ end, \ st^{'}, \ E_l^{'} \rangle}	
	\]\newline
	\[	
	[ATOM-COMPLETE]	
	\dfrac{E_l, \ E_f, \ E_a \ \vdash \langle S, \ st, \ E_l \rangle \ \Rightarrow \ \langle st^{'}, \ E_l^{'} \rangle}{E_e, \ E_f, \ E_a \ \vdash \ \langle atom \ S \ end, \ st, \ E_l \rangle \ \Rightarrow \ \langle st^{'}, \ E_l^{'} \rangle}	
	\]
	
In the formation rules section the formation rule "Game", describes the interaction between the battlefield and the program. The "F" in Game is the battlefield configuration. 
	
F have to states in our language, either an event is raised or not. If no events are raised, F stays the same. If an event has been raise \begin{math} F \rightarrow_{event(p)} F^{'} \end{math}, which is an transition from the battlefield configuration. It is possible for the events to be raised with an parameter p. 

	\[	
	[PARA-F]	
	\dfrac{E_e, \ E_f, \ E_a \ \vdash \ \langle S, \ st, \ E_l \rangle \ \Rightarrow_S \ \langle S^{'}, \ st^{'}, \ E_l^{'}}{E_e, \ E_f, \ E_a \ \vdash \ \langle S \ || \ F, \ st, \ E_l \rangle \ \Rightarrow_Para \ \langle S^{'} \ || \ F, \ st^{'}, \ E_l^{'} \rangle}	
	\]\newline
	\[	
	[PARA-F^{'}]	
	\dfrac{F \ \rightarrow_{event(p)} \ F^{'}}{E_e, \ E_f, \ E_a \ \vdash \ \langle S \ || \ F, \ st, \ E_l \rangle \ \Rightarrow_Para \ \langle atom \ S^{'} \ end; \ S \ || \ F^{'}, \ st, \ E_l \rangle}	
	\]
	\begin{math}		
		Where \ S \ is \ not \ of \ the \ form \ S_1 \ end; \ S_2 \newline		
		E_e(event) \ = \ (begin \ D_v \ S^{''} \ end, \ x, \ E_v) \newline		
		S^{'} \ = \ begin \ D_v \ (x \ := \ p; \ S^{''} \ end)		
	\end{math}
	
	\[
	[COMP-PARTLY]
	\dfrac{\langle S_1, \ st, \ E_l \rangle \ \Rightarrow \ \langle S_1^{‘}, \ st^{‘}, \ E_l^{‘}}{\langle S_1; \ S_2, \ st, \ E_l \rangle \ \Rightarrow \ \langle S_1^{‘}; \ S_2, \ st^{‘}, \ E_l{‘}}
	\]\newline
	\[
	[COMP-COMPLETE]
	\dfrac{\langle S_1, \ st, \ E_l \rangle \ \Rightarrow \ \langle \ st^{‘}, \ E_l^{‘}}{\langle S_2, \ st, \ E_l \rangle \ \Rightarrow \ \langle S_1^{‘}; \ S_2, \ st^{‘}, \ E_l{‘}}
	\]\newline
	\[
	[ASSIGN]
	\dfrac{}{\langle x \ := \ a, \ st, \ E_l \rangle \ \Rightarrow \ (st[l \mapsto v], \ E_l)}
	\]
	\begin{math}
		Where
		\qquad E_l = (E_v) : E_l^{‘}
		\qquad l = E_v(x)
		\qquad E_v, st \vdash a \rightarrow_a v
	\end{math}
	
	\[
	[IF-^{\top}_e]
	\dfrac{\langle if \ e \ B_1 \ else \ B_2, \ st, \ E_l \rangle \ \Rightarrow \ \langle B_1, \ st, \ E_l \rangle}{}
	\]
	\begin{math}
		If E_v, st \vdash e \rightarrow_e \top \quad where E_l = (E_v) : E_l{‘}
	\end{math}
	
	\[
	[IF-^{\bot}_e]
	\dfrac{\langle if \ e \ B_1 \ else \ B_2, \ st, \ E_l \rangle \ \Rightarrow \ \langle B_2, \ st, \ E_l \rangle}{}
	\]
	\begin{math}
		If E_v, st \vdash e \rightarrow_e \bot \quad where E_l = (E_v) : E_l{‘}
	\end{math}
	
	\[
	[CALL-FUNC]
	\dfrac{}{ \langle call \ f(x_1 … x_n), \ st, \ E_l \rangle \ \Rightarrow \ \langle active \ S \ end, \ st, \ (E_v^{‘}, E_f^{‘}) : E_l \rangle}
	\]
	\begin{math}
		Where 
		\qquad E_f(f) = (S, E_v^{‘}, E_f^{‘})
		\qquad E_l = (E_v) : E_l^{‘}
	\end{math}
	
	\[
	[CALL-F-REC]
	\dfrac{}{ \langle call \ f, \ st, \ E_l \rangle \ \Rightarrow \ \langle active \ S \ end, \ st, \ (E_v^{‘}, \ E_f^{‘}[f \mapsto (S, \ E_v^{‘}, \ E_f^{‘}] : E_l \rangle}
	\]
	\begin{math}
		Where
		\qquad E_f(f) = (S, E_v^{‘}, E_f^{‘}) 
		\qquad E_l = (E_v) : E_l^{‘}
	\end{math}
	
	\[
	[CALL-ACT]
	\dfrac{}{ \langle call \ a(x_1 … x_n), \ st, \ E_l \rangle \ \Rightarrow \ \langle active \ S \ end, \ st, \ (E_v^{‘}, E_a^{‘}) : E_l}
	\]
	\begin{math}
		Where
		\qquad E_a(a) = (S, E_v^{‘}, E_a^{‘})
		\qquad E_l = (E_v) : E_l^{‘}
	\end{math}
	\[
	[CALL-A-REC]
	\dfrac{}{ \langle call \ f, \ st, \ E_l \rangle \ \Rightarrow \ \langle active \ S \ end, \ st, \ (E_v^{‘}, \ E_f^{‘}[f \mapsto (S, \ E_v^{‘}, \ E_f^{‘}] : E_l \rangle}
	\]
	\begin{math}
		Where
		\qquad E_a(a) = (S, E_v^{‘}, E_a^{‘})
		\qquad E_l = (E_v) : E_l^{‘}		
	\end{math}
	
	\[
	[IF-^{\top}_e]
	\dfrac{\langle if \ e \ B_1 \ else \ B_2, \ st, \ E_l \rangle \ \Rightarrow \ \langle B_1, \ st, \ E_l \rangle}{}
	\]
	\begin{math}
		If E_v, st \vdash e \rightarrow_e \top \quad where E_l = (E_v) : E_l{‘}
	\end{math}
	
	\[
	[IF-^{\bot}_e]
	\dfrac{\langle if \ e \ B_1 \ else \ B_2, \ st, \ E_l \rangle \ \Rightarrow \ \langle B_2, \ st, \ E_l \rangle}{}
	\]
	\begin{math}
		If E_v, st \vdash e \rightarrow_e \bot \quad where E_l = (E_v) : E_l{‘}
	\end{math}
	
	\[
	[REPEAT]
	\dfrac{}{\langle repeat \ while(e) \ B, \ st, \ E_l \rangle \ \Rightarrow \ \langle if \ e \ (B; \ repeat \ while(e) \ B) \ else \ skip, \ st, \ E_l \rangle}
	\]\newline
	\[
	[SKIP]
	\dfrac{}{\langle skip, \ st, \ E_l \rangle \ \Rightarrow \ (st, \ E_l)}
	\]\newline
	\[
	[RETURN]	
	\dfrac{E_v, \ E_a, \ E_f \ \vdash \ \langle e, \ st \rangle \ \rightarrow_e \ (v, \ st^{'})}{E_f, \ E_a \ \vdash \ \langle return \ e, \ st, \ E_l \rangle \ \Rightarrow \ (st^{'}[r \ \mapsto \ v], \ E_l)}
	\quad Where	
	\qquad E_l = E_v : E_l^{'}	
	\]\newline

	
\subsection{Expressions}
	\[	
	[PAR-EXPR] \quad	
	\dfrac{E_v, \ E_a, \ E_f \vdash \langle e, \ st \rangle \rightarrow_e (v, \ st^{'}}{E_v, \ E_a, \ E_f \vdash \langle (e), \ st \rangle \rightarrow_e \ (v, \ st^{'})}	
	\]

\subsection{Arithmetic expressions}
	\[
	[ASM-EXPR] \quad	
	\dfrac{E_v, \ E_a, \ E_f \vdash \langle a_1, \ st \rangle \rightarrow_{e} (v_1, st^{''}) \qquad E_v, \ E_a, \ E_f \vdash \langle a_2, \ st^{''}\rangle \rightarrow_e (v_2, \ st^{'})}{E_v, \ E_a, \ E_f \vdash \ \langle a_1 \ op \ a_2, st \rangle \ \rightarrow_{e} (v, st^{'})}	
	\]
	\begin{math}		
		op \ \in \ \{ \ +, \ -, \ * \ \} \newline		
		v \ = \ v_1 \ op \ v_2		
	\end{math}

	\[
	[DM-EXPR] \quad
	\dfrac{E_v, \ E_a, \ E_f \vdash \ \langle a_1, \ st \rangle \ \rightarrow_{e} (v_1, st^{''}) \qquad E_v, \ E_a, \ E_f \vdash \ \langle a_2, \ st^{''} \rangle \ \rightarrow_e (v_2, \ st^{'})}{E_v, \ E_a, \ E_f \vdash \ \langle a_1 \ op \ a_2, st \rangle \ \rightarrow_{e} (v, st^{'})}	
	\]
	\begin{math}		
		op \ \in \ \{ \ /, \ \% \ \} \ | v \not= \ 0 \newline 		
		v \ = \ v_1 \ op \ v_2		
	\end{math} 
	
\subsection{Boolean expressions}
Most of the boolean expressions look alike even with the value of true/false, except the AND/OR expressions. The constructs of AND/OR uses short-circuiting, since in the OR expression if the first value is evaluated to true, there is no need to evaluate the second part of the expression.
	\[
	[EXPR^{IS=}_{\ \top}] \quad	
	\dfrac{E_v, \ E_a, \ E_f \vdash \ \langle e_1, \ st \rangle \ \rightarrow_e (v_1, st^{''}) \qquad E_v, \ E_a, \ E_f \vdash \ \langle e_2, \ st^{''} \rangle \ \rightarrow_e (v_2, \ st^{'})}{E_v, \ E_a, \ E_f \ \vdash \ \ \langle e_1 \ IS= e_2, \ st \rangle \ \ \rightarrow_e (\top, \ st^{'})}	
	\]
	\begin{math}		
		v_1 \ = \ v_2 		
	\end{math}	
	\[
	[EXPR^{IS=}_{\ \bot}] \quad	
	\dfrac{E_v, \ E_a, \ E_f \vdash \ \langle e_1, \ st \ \rangle \rightarrow_e (v_1, st^{''}) \qquad E_v, \ E_a, \ E_f \vdash \ \langle e_2, \ st^{''} \rangle \ \rightarrow_e (v_2, \ st^{'})}{E_v, \ E_a, \ E_f \ \vdash \ \langle e_1 \ IS= e_2, \ st \rangle \ \rightarrow_e (\bot, \ st^{'})}	
	\]
	\begin{math}		
		v_1 \ \not= \ v_2		
	\end{math}
	\[	
	[EXPR^{NOT=}_{\ \top}] \quad	
	\dfrac{E_v, \ E_a, \ E_f \vdash \ \langle e_1, \ st \rangle \ \rightarrow_e (v_1, st^{''}) \qquad E_v, \ E_a, \ E_f \vdash \ \langle e_2, \ st^{''} \rangle \ \rightarrow_e (v_2, \ st^{'})}{E_v, \ E_a, \ E_f \ \vdash \ \langle e_1 \ NOT= e_2, \ st \rangle \ \rightarrow_e (\top, \ st^{'})}	
	\]	
	\begin{math}		
		v_1 \ \not= \ v_2		
	\end{math} 
	\[
	[EXPR^{NOT=}_{\ \bot}] \quad	
	\dfrac{E_v, \ E_a, \ E_f \vdash \ \langle e_1, \ st \rangle \ \rightarrow_e (v_1, st^{''}) \qquad E_v, \ E_a, \ E_f \vdash \ \langle e_2, \ st^{''} \rangle \ \rightarrow_e (v_2, \ st^{'})}{E_v, \ E_a, \ E_f \ \vdash \ \langle e_1 \ NOT= e_2, \ st \rangle \ \rightarrow_e (\bot, \ st^{'})}	
	\]	
	\begin{math}		
		v_1 \ = \ v_2		
	\end{math} 
	\[	
	[EXPR^{\ >}_{\ \top}] \quad	
	\dfrac{E_v, \ E_a, \ E_f \vdash <e_1, \ st> \rightarrow_e (v_1, st^{''}) \qquad E_v, \ E_a, \ E_f \vdash <e_2, \ st^{''} \rightarrow_e (v_2, \ st^{'})}{E_v, \ E_a, \ E_f \ \vdash \ <e_1 \ > e_2, \ st> \ \rightarrow_e (\top, \ st^{'})}	
	\]
	\begin{math}		
		v_1 \ > \ v_2		
	\end{math} 
	\[	
	[EXPR^{\ >}_{\ \bot}] \quad	
	\dfrac{E_v, \ E_a, \ E_f \vdash \ \langle e_1, \ st \rangle \ \rightarrow_e (v_1, st^{''}) \qquad E_v, \ E_a, \ E_f \vdash \ \langle e_2, \ st^{''} \rangle \ \rightarrow_e (v_2, \ st^{'})}{E_v, \ E_a, \ E_f \ \vdash \ \langle e_1 \ > e_2, \ st \rangle \ \rightarrow_e (\bot, \ st^{'})}
	\]
	\begin{math}		
		v_1 \ \not> \ v_2		
	\end{math} 
	\[	
	[EXPR^{\ <}_{\ \top}] \quad	
	\dfrac{E_v, \ E_a, \ E_f \vdash \ \langle e_1, \ st \rangle \ \rightarrow_e (v_1, st^{''}) \qquad E_v, \ E_a, \ E_f \vdash \ \langle e_2, \ st^{''} \rangle \ \rightarrow_e (v_2, \ st^{'})}{E_v, \ E_a, \ E_f \ \vdash \ \langle e_1 \ < e_2, \ st \rangle \ \rightarrow_e (\top, \ st^{'})}
	\]
	\begin{math}		
		v_1 \ < \ v_2		
	\end{math} 
	\[	
	[EXPR^{\ <}_{\ \bot}] \quad	
	\dfrac{E_v, \ E_a, \ E_f \vdash \ \langle e_1, \ st \rangle \ \rightarrow_e (v_1, st^{''}) \qquad E_v, \ E_a, \ E_f \vdash \ \langle e_2, \ st^{''} \rangle \ \rightarrow_e (v_2, \ st^{'})}{E_v, \ E_a, \ E_f \ \vdash \ \langle e_1 \ < e_2, \ st \rangle \ \rightarrow_e (\bot, \ st^{'})}	
	\]
	\begin{math}		
		v_1 \ \not< \ v_2		
	\end{math}
	\[	
	[EXPR^{\ >=}_{\ \ \top}] \quad	
	\dfrac{E_v, \ E_a, \ E_f \vdash \ \langle e_1, \ st \rangle \ \rightarrow_e (v_1, st^{''}) \qquad E_v, \ E_a, \ E_f \vdash \ \langle e_2, \ st^{''} \rangle \ \rightarrow_e (v_2, \ st^{'})}{E_v, \ E_a, \ E_f \ \vdash \ \langle e_1 \ >= e_2, \ st \rangle \ \rightarrow_e (\top, \ st^{'})}	
	\]
	\begin{math}		
		v_1 \ \ge \ v_2		
	\end{math} 
	\[	
	[EXPR^{\ >=}_{\ \ \bot}] \quad	
	\dfrac{E_v, \ E_a, \ E_f \vdash \ \langle e_1, \ st \rangle \ \rightarrow_e (v_1, st^{''}) \qquad E_v, \ E_a, \ E_f \vdash \ \langle e_2, \ st^{''} \rangle \ \rightarrow_e (v_2, \ st^{'})}{E_v, \ E_a, \ E_f \ \vdash \ \langle e_1 \ >= e_2, \ st \rangle \ \rightarrow_e (\bot, \ st^{'})}	
	\]
	\begin{math}		
		v_1 \ \ngeq \ v_2		
	\end{math}
	\[
	[EXPR^{\ <=}_{\ \ \top}] \quad
	\dfrac{E_v, \ E_a, \ E_f \vdash \ \langle e_1, \ st \rangle \ \rightarrow_e (v_1, st^{''}) \qquad E_v, \ E_a, \ E_f \vdash \ \langle e_2, \ st^{''} \rangle \ \rightarrow_e (v_2, \ st^{'})}{E_v, \ E_a, \ E_f \ \vdash \ \langle e_1 \ <= e_2, \ st \rangle \ \rightarrow_e (\top, \ st^{'})}
	\]
	\begin{math}		
		v_1 \ \le \ v_2		
	\end{math} 
	\[	
	[EXPR^{\ <=}_{\ \ \bot}] \quad	
	\dfrac{E_v, \ E_a, \ E_f \vdash \ \langle e_1, \ st \rangle \ \rightarrow_e (v_1, st^{''}) \qquad E_v, \ E_a, \ E_f \vdash \ \langle e_2, \ st^{''} \rangle \ \rightarrow_e (v_2, \ st^{'})}{E_v, \ E_a, \ E_f \ \vdash \ \langle e_1 \ <= e_2, \ st \rangle \ \rightarrow_e (\bot, \ st^{'})}	
	\]
	\begin{math}		
		v_1 \ \nleq \ v_2		
	\end{math}
	\[	
	[EXPR^{\ \ !}_{\ \ \top}] \quad	
	\dfrac{E_v, \ E_a, \ E_f \vdash \ \langle e, \ st \rangle \ \rightarrow_e (\top, st^{'})}{E_v, \ E_a, \ E_f \ \vdash \ \langle !e, \ st \rangle \ \rightarrow_e (\top, \ st^{'})}	
	\] \newline
	\[	
	[EXPR^{\ \ !}_{\ \ \bot}] \quad	
	\dfrac{E_v, \ E_a, \ E_f \vdash \ \langle e, \ st \rangle \ \rightarrow_e (\bot, st^{'})}{E_v, \ E_a, \ E_f \ \vdash \ \langle !e, \ st \rangle \ \rightarrow_e (\bot, \ st^{'})}	
	\]\newline
	\[	
	[EXPR^{\ AND}_{\ \ \top}] \quad	
	\dfrac{E_v, \ E_a, \ E_f \ \vdash \ \langle e_1, \ st \rangle \ \rightarrow_e \ (\top, \ st^{''}) \qquad E_v, \ E_a, \ E_f \ \vdash \ \langle e_2, \ st^{''} \ \rightarrow_e \ (v, st^{'})}{E_v, \ E_a, \ E_f \ \vdash \ <e_1 \ AND \ e_2, \ st> \ \rightarrow_e (v, \ st^{'})}	
	\] \newline
	\[	
	[EXPR^{\ AND}_{\ \ \bot}] \quad	
	\dfrac{E_v, \ E_a, \ E_f \vdash <e_1, \ st> \rightarrow_e (\bot, \ st^{'}) }{E_v, \ E_a, \ E_f \ \vdash \ <e_1 \ AND \ e_2, \ st> \ \rightarrow_e (\bot, \ st^{'})}	
	\]\newline
	\[	
	[EXPR^{\ OR}_{\ \ \top}] \quad	
	\dfrac{E_v, \ E_a, \ E_f \vdash <e_1, \ st> \rightarrow_e (\top, \ st^{'}) }{E_v, \ E_a, \ E_f \ \vdash \ <e_1 \ OR \ e_2, \ st> \ \rightarrow_e (\top, \ st^{'})}	
	\]\newline
	\[	
	[EXPR^{\ OR}_{\ \ \bot}] \quad	
	\dfrac{E_v, \ E_a, \ E_f \ \vdash \ \langle e_1, \ st \rangle \ \rightarrow_e \ (\bot, \ st^{''}) \qquad E_v, \ E_a, \ E_f \ \vdash \ \langle e_2, \ st^{''} \ \rightarrow_e \ (v, st^{'})}{E_v, \ E_a, \ E_f \ \vdash \ <e_1 \ OR \ e_2, \ st> \ \rightarrow_e (v, \ st^{'})}
	\]
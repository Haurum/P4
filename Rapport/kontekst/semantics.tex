\chapter{Semantics}
This chapter provides a formal description of the language semantics. The chapter uses techniques from the book \textit{Transitions and Trees} \citep{Huttel}.

\section{Syntactic categories}	
To simplify the presentation of the semantics of our language, syntactic categories have been used. The syntactic categories are based on the grammar found in section \ref{sec:Grammar}. A collection of meta variables are presented in the following paragraphs which will be used throughout the chapter to present the operational semantics.

\begin{math}		
e \in \textbf{Expr} - Expressions \newline		
S \in \textbf{Stmt} - Statements\newline	
n \in \textbf{Num} - Numerals\newline	
b \in \textbf{Bool} - Boolean\ literals\newline		
B \in \textbf{Block} - Blocks \newline	
tx \in \textbf{String} - String\ literals\newline	
t \in \textbf{Type} - Num,\ Bool\ and\ String\ types\newline	
x \in \textbf{var} - Variable \ name \newline	
f \in \textbf{Func} - Function \ name \newline	
a \in \textbf{Act} - Action \ name \newline	
D_a \in \textbf{ActDcl} - Action\ declaration\newline
D_f \in \textbf{FuncDcl} - Function\ declaration\newline
D_v \in \textbf{VarDcl} - Gobal\ variable\ declaration\newline
F \in \textbf{Battlefield} - Battlefield\ configuration\newline
\end{math}

\section{Formation rules}
These rules have the purpose of defining how to format code in the language. It doesn't define how things works, this will be described later in the semantics.

\begin{math}	
B ::= \{ \ D_v \ S \ \}	
\newline	
S ::= \ skip \ | \ x \ = \ a \ | \ B \ | \ call \ a(e_1 ... , e_n) \ | \ return \ e \ | \ S_1; \ S_2 \ | \ if \ e \ B_1 \ else \ B_2 \
\newline  | \ repeat \ while \ e \ B \ | \ active \ S \ end \ | \ atom \ S \ end
\newline
a ::= \ action \ a(x_1 ... x_n) \ is \ B \ D_a \ | \epsilon 	
\newline	
f ::= \ function \ f(x_1 ... x_n) \ is \ B \ D_f \ | \epsilon 
\newline
D_v ::= \ \tau \ x \ D_v \ | \epsilon 
\newline
e ::= \ ( \ e \ ) \ | \ call \ f(x_1 ... x_n) \ | \ e_1 op e_2 \ | \ !e \ | \ \epsilon
\newline
Game ::= \ S \ || \ F
\newline
op \in {+, -, *, /, \%, =, \neq, >, <, \nleq, \ngeq, \not>, \nless, \ge, \le, AND, OR}
\end{math}

\section{Semantic functions}
The purpose of the semantics functions to interpret syntactic elements to semantic elements. The semantic functions used in our language will be described in this section. 

\subsection{Numeral literals}	
The numbers literal in our language will be Numerals, which will be interpreted to real numbers by means of the semantic function: 

\begin{math}
\mathcal{N}: \textbf{Num} \rightarrow \mathbb{R}
\end{math}

Using this function, numerals as 
\begin{math}
\mathcal{N}
\end{math}[\underline{5}] and 
\begin{math}
\mathcal{N}		
\end{math}[\underline{5.36}] will be mapped to the corresponding values 5 and 5.36. 

\subsection{String literals}
A String literal is a sequence of symbols and characters in UTF-8(Unicode Transformation Format 8-bit) except the delimiter ("). The sequence of symbols and characters must be within the delimiter, for example "Hello world!". \newline
String literal are interpreted as string elements Text, by means of the following:
\begin{math}		
\tau : \textbf{String} \ \rightarrow \ \textbf{StringL} \newline	
As \ an \ example: \ \tau("Insert \ text \ here") \ \rightarrow \ Insert \ text \ here		
\end{math}

\subsection{Boolean literals}
The Boolean literals depict whether the expression is evaluated to True or False. The semantic value for true is \begin{math} \top \end{math}, and false is \begin{math} \bot. \end{math}

\section{Environment store model}
The environment store model describes how variables are bound to a storage during the execution of programs. The variables are bound to a storage-call, where the storage-cell contains the value of the variable. Storage-cells are also called \textit{locations}, the set of locations is denoted \textbf{Loc}.

The variable environment is a function that will tell what storage location a variable is bound to. Variable environments is the set of partial functions from the variable to their location, as shown in the following equation:

\begin{math} 		
E_v \in \boldsymbol{EnvV} = \boldsymbol{Var} \cup \{ next \} \rightharpoonup \ \boldsymbol{Loc}		
\end{math}

We assume the \textit{next} element does not belong to \textit{var} and will be used to point to the next available location in memory.

Let l be an arbitrary element of \textbf{Loc} and assume that \textbf{Loc} = \begin{math} \mathbb{N}. \end{math} The function \textbf{new} found below, will return the successor for every location by defining \textbf{new} as: \textbf{new} l = l + 1.

\begin{math}		
\boldsymbol{new} : \ \boldsymbol{Loc} \ \rightarrow \ \boldsymbol{Loc}		
\end{math}

Let \begin{math}E_v\end{math} be an arbitrary element in \textbf{EnvV}, we denote by \begin{math} E_v[x \mapsto l] \end{math} the variable environment \begin{math}E_v^{'} \end{math} defined as follows:
\[ E^{'}_v(y) =	
\begin{cases}
E_v(y) & \quad \text{if } y \not= x\\
l & \quad \text{if } y = x\\	
\end{cases}	
\]
\begin{math}Env_v[x \mapsto v] \end{math} represents the update of \begin{math} x \ \in \ var \ to \ l \ \in \ Loc.\end{math} 

A partial function in \textbf{St} is defined below with an arbitrary element st.

\begin{math}		
st \ \in \boldsymbol{St} \ = \ \boldsymbol{Loc} \rightharpoonup \ \boldsymbol{Value}		
\end{math}

For updating the store st \begin{math} \in \end{math} \textbf{St} the update notation \begin{math} sto[l \mapsto v] \end{math} will be used. \begin{math} sto[l \mapsto v] \end{math} will denote the store function \begin{math}sto^{'} \end{math} defined as follows:
\[ sto^{'}(l^{'}) =	
\begin{cases}	
sto(l) & \quad \text{if } l^{'} \not= l\\	
v & \quad \text{if } l^{'} = l\\	
\end{cases}	
\]

\section{Operational semantic}
This section has the purpose of formally describing the language. It consists of semantic rules that give context to a program. \newline
The semantics is described with big-step semantics due to the intuitiveness of this kind of semantics. \newline
The environments during the operational semantics will be denoted with E. For simplicity the environments \begin{math} E_v, E_a, E_f \end{math} will now be denoted as \begin{math} E_{vaf} \end{math} and \begin{math} E_a, E_f \end{math} will be denoted as \begin{math} E_{af} \end{math}.

\subsection{Blocks}
\[	
[BLOCK-ENTER]	
\dfrac{\langle D_v, \ E_v \rangle \ \rightarrow_{Dv} \ E_v^{'}}{E_{eaf} \ \vdash \ \langle active \ S \ end, \ st, \ E_l \rangle \ \Rightarrow \ \langle active \ S \ end, \ st, \ E_v^{'} \ : \ E_l}	
\]
\[	
[BLOCK-CONTINUE]	
\dfrac{E_{eaf} \ \vdash \ \langle S, \ st, \ E_l \rangle \ \Rightarrow \ \langle S^{'}, \ st^{'}, \ E_l^{'}}{E_{eaf} \ \vdash \ \langle active \ S \ end, \ st, \ E_l \rangle \ \Rightarrow \ \langle active \ S^{'} \ end, \ st^{'}, \ E_l^{'} \rangle}	
\]
\[	
[BLOCK-EXIT]	
\dfrac{E_{eaf} \ vdash \ \langle S, \ st, \ E_l \rangle \ \Rightarrow \ \langle st^{'}, \ E_v \ : \ E_l^{'} \rangle}{E_{eaf} \ \vdash \ \langle active \ S \ end, \ st, \ E_l \ \rangle \ \Rightarrow \ \langle st^{'}, \ E_l^{'} \rangle}	
\]

\subsubsection{Declarations}	
\[	
[EMPTY-VARDCL] \quad	
\dfrac{}{\langle\epsilon, \ E_v\rangle \rightarrow_{Dv} E_v}	
\]\newline
\[	
[VARDCL] \quad	
\dfrac{ \langle D_v, \ E_v[x \mapsto l][next \mapsto new(l)]\rangle \ \rightarrow_{Dv} \ E_v^{'}}{ \langle \ \tau \ x; \ D_v,\ E_v \rangle \rightarrow_{Dv} \ E_v^{'}} \newline	
\qquad \ \ l = E_v(next)	
\]\newline
\[	
[EMPTY-ACTDCL] \quad	
\dfrac{}{E_v \ \vdash \ \langle \epsilon, E_a \rangle \ \rightarrow_{Da} \ E_a}	
\]\newline
\[	
[ACTDCL] \quad	
\dfrac{E_v \ \vdash \langle D_a, \ E_a[a \ \mapsto (B, \ x_1 .. x_n \ E_v)] \rangle \rightarrow_{Da} \ E_a^{'}}{E_v \ \vdash \ \langle Action \ a(x_1 .. x_n) \ is \ B; \ D_a, \ E_a \rangle \ \rightarrow_{Da} \ E_a^{'}}	
\]\newline
\[	
[EMPTY-FUNCDCL] \quad	
\dfrac{}{E_v \ \vdash \langle \epsilon, \ D_f \rangle \ \rightarrow_{Df} \ E_f}	
\]\newline
\[	
[FUNCDCL] \quad	
\dfrac{E_v \vdash \langle D_f, \ E_f[f \ \mapsto (B, \ x_1 .. x_n, \ E_v)] \rangle \rightarrow_{Df} E_f^{'}}{E_v \vdash \ \langle Function \ f(x_1 .. x_n) \ is \ B; \ D_f, \ E_f \rangle \ \rightarrow_{Df} \ E_f^{'}}	
\]

\subsection{Statements}
The interaction with the battlefield will not be explained since it is out of the scope of the project. The interaction that a battlefield configuration may have with a statement running in parallel will be explained.

The configuration of the semantic statements are:
\begin{itemize}
	\item \begin{math} \langle S, \ st, \ E_l \rangle \end{math}, which are intermediate configurations where S is a statement that may contain atomic-encapsulations or parallels.
	\item \begin{math} \langle st, \ E_l \rangle \end{math} is terminal configurations.	
\end{itemize}

The transition system for statements is:
\begin{math}		
((\boldsymbol{Stmt} \ \times \boldsymbol{St} \ \times \boldsymbol{E_l}) \ \cup \ (\boldsymbol{St} \ \times \ \boldsymbol{E_l}), \Rightarrow, \ (\boldsymbol{St} \ \times \ \boldsymbol{E_l}))		
\end{math}

The set of run time stacks is:
\textbf{Envl} = \textbf{EnvV}\begin{math}		
^*		
\end{math}

We will denote by \begin{math} E_l \end{math} a generic element in EvCxt
For simplicity we will now denote \begin{math} E_l, E_a, E_f \end{math} as \begin{math} E_{laf} \end{math} and \begin{math} E_e, E_a, E_f \end{math} as \begin{math} E_{eaf} \end{math} and \begin{math} E_v, E_f \end{math} as \begin{math} E_{vf} \end{math}, and lastly the environments \begin{math} E_v, E_e, E_a, E_f \end{math} as \begin{math} E_{veaf} \end{math}.

\[	
[ATOM-CONTINUE]	
\dfrac{E_{eaf} \ \vdash \langle S, \ st, \ E_l \rangle \ \Rightarrow \ \langle S^{'}, \ st^{'}, \ E_l^{'} \rangle}{E_{eaf} \ \vdash \ \langle atom \ S \ end, \ st, \ E_l \rangle \ \Rightarrow \ \langle atom \ S^{'} \ end, \ st^{'}, \ E_l^{'} \rangle}	
\]\newline
\[	
[ATOM-EXIT]	
\dfrac{E_{eaf} \ \vdash \langle S, \ st, \ E_l \rangle \ \Rightarrow \ \langle st^{'}, \ E_v \ : \ E_l^{'} \rangle}{E_{eaf} \ \vdash \ \langle atom \ S \ end, \ st, \ E_l \rangle \ \Rightarrow \ \langle st^{'}, \ E_l^{'} \rangle}	
\]

In the formation rules section the formation rule "Game", describes the interaction between the battlefield and the program. The "F" in Game is the battlefield configuration. 

F have to states in our language, either an event is raised or not. If no events are raised, F stays the same. If an event has been raise \begin{math} F \rightarrow_{event(p)} F^{'} \end{math}, which is an transition from the battlefield configuration. It is possible for the events to be raised with an parameter p. 

\[	
[PARA-PROG]	
\dfrac{E_{eaf} \ \vdash \ \langle S, \ st, \ E_l \rangle \ \Rightarrow_S \ \langle S^{'}, \ st^{'}, \ E_l^{'} \rangle}{E_{eaf} \ \vdash \ \langle S \ || \ F, \ st, \ E_l \rangle \ \Rightarrow \ \langle S^{'} \ || \ F, \ st^{'}, \ E_l^{'} \rangle}
\]\newline
\[	
[PARA-EVENT]	
\dfrac{F \ \rightarrow_{event(p)} \ F^{'} \quad \langle D_v, \ E_v[x \mapsto l][next \mapsto new(l)] \rangle \ \rightarrow_{Dv} \ E_v^{'}}{E_{eaf} \ \vdash \ \langle S \ || \ F, \ st, \ E_l \rangle \ \Rightarrow \ \langle atom \ S^{'} \ end; \ S \ || \ F^{'}, \ st[l \mapsto p], \ E_v^{'} \ : \ E_l \rangle}	
\]
\begin{math}		
Where \ S \ is \ not \ of \ the \ form \ S_1 \ end; \ S_2 \newline
E_v \ = \ E_v \ : \ E_l^{'}, \ and \ l \ = \ E_v(next) \newline	
E_e(event) \ = \ (begin \ D_v \ S^{'} \ end, \ x, \ E_v) \newline	
\end{math}
\[	
[PARA-QUIT]	
\dfrac{F \ \rightarrow_{event(p)} \ F^{'} \quad \langle D_v, \ E_v[x \mapsto l][next \mapsto new(l)] \rangle \ \rightarrow_{Dv} \ E_v^{'}}{E_{eaf} \ \vdash \ \langle S \ || \ F, \ st, \ E_l \rangle \ \Rightarrow \ \langle active \ S^{'} \ end, \ st[l \mapsto p], \ E_v^{'} \ : \ E_l \rangle}	
\]
\begin{math}		
	Where \ S \ is \ not \ of \ the \ form \ S_1 \ end; \ S_2 \newline
	E_v \ = \ E_v \ : \ E_l^{'}, \ and \ l \ = \ E_v(next) \newline	
	E_e(quit) \ = \ (begin \ D_v \ S^{'} \ end, \ x, \ E_v) \newline	
\end{math}

\[
[COMP-PARTLY]
\dfrac{E_{eaf} \ \vdash \ \langle S_1, \ st, \ E_l \rangle \ \Rightarrow \ \langle S_1^{'}, \ st^{'}, \ E_l^{'} \rangle}{E_{eaf} \ \vdash \ \langle S_1; \ S_2, \ st, \ E_l \rangle \ \Rightarrow \ \langle S_1^{'}; \ S_2, \ st^{'}, \ E_l{'} \rangle}
\]\newline
\[
[COMP-COMPLETE]
\dfrac{E_{eaf} \ \vdash \ \langle S_1, \ st, \ E_l \rangle \ \Rightarrow \ \langle \ st^{'}, \ E_l^{'} \rangle}{E_{eaf} \ \vdash \ \langle S_2, \ st, \ E_l \rangle \ \Rightarrow \ \langle S_1^{'}; \ S_2, \ st^{'}, \ E_l{'} \rangle}
\]\newline
\[
[ASSIGN]
\dfrac{E_{veaf} \ \vdash \ \langle e, \ st \rangle \ \rightarrow_e \ (v, \ st^{'})}{E_{eaf} \ \vdash \ \langle x \ = \ e, \ st, \ E_l \rangle \ \Rightarrow \ (st^{'}[l \mapsto v], \ E_l)}
\]
\begin{math}
Where
\qquad E_l = E_v : E_l^{‘}
\qquad l = E_v(x)
\end{math}

\[
[IF-^{\top}_e]
\dfrac{E_{vf} \ \vdash \ \langle e, \ st \rangle \ \rightarrow_e \ (\top, \ st^{'})}{E_{eaf} \ \vdash \ \langle if \ e \ B_1 \ else \ B_2, \ st, \ E_l \rangle \ \Rightarrow \ \langle B_1, \ st^{'}, \ E_l \rangle}
\]
\begin{math}
Where \quad E_l = E_v : E_l{‘}
\end{math}

\[
[IF-^{\bot}_e]
\dfrac{E_{vf} \ \vdash \ \langle e, \ st \rangle \ \rightarrow_e \ (\bot, \ st^{'})}{E_{eaf} \ \vdash \ \langle if \ e \ B_1 \ else \ B_2, \ st, \ E_l \rangle \ \Rightarrow \ \langle B_2, \ st^{'}, \ E_l \rangle}
\]
\begin{math}
Where \quad E_l = E_v : E_l{‘}
\end{math}

\[
[CALL-ACT]
\dfrac{E_{veaf} \ \vdash \ \langle e_i, \ st_i \rangle \ \rightarrow_e \ (v_i, \ st_{i+1}^{'}) \quad for \ i \ \in \ {1 ... n}}{E_{eaf} \ \vdash \ \langle call \ a(e_1 … e_n), \ st, \ E_l \rangle \ \Rightarrow \ \langle active \ B \ end, \ st^{'}, \ E_v^{''} \ : \ E_l}
\]
\begin{math}
Where
\qquad st_1 = st, E_l = E_v : E_l, E_a(a) = (B, (x_1 ... x_n), E_v^{'}) \newline
\qquad l_1 = E_v(next) and, l_{j+1} = new(l_j) for j \in {1 ... n} \newline
\qquad E_v^{''} = E_v^{'}[x_1 \mapsto l_1] ... [x_n ... l_n][next \mapsto l_{n+1}] \newline
\qquad st^{'} = st_{n+1}[l_1 \mapsto v_1] ... [l_n \mapsto v_n]
\end{math}

\[
[REPEAT]
\dfrac{}{E_{eaf} \ \vdash \ \langle repeat \ while(e) \ B, \ st, \ E_l \rangle \ \Rightarrow \ \langle if \ e \ (B; \ repeat \ while(e) \ B) \ else \ skip, \ st, \ E_l \rangle}
\]\newline
\[
[SKIP]
\dfrac{}{\langle skip, \ st, \ E_l \rangle \ \Rightarrow \ (st, \ E_l)}
\]\newline
\[
[RETURN]	
\dfrac{E_{veaf} \ \vdash \ \langle e, \ st \rangle \ \rightarrow_e \ (v, \ st^{'})}{E_{eaf} \ \vdash \ \langle return \ e, \ st, \ E_l \rangle \ \Rightarrow \ (st^{'}[r \ \mapsto \ v], \ E_l)}
\quad Where	
\qquad E_l = E_v : E_l^{'}	
\]\newline


\subsection{Expressions}
\[	
[PAR-EXPR] \quad	
\dfrac{E_{veaf} \vdash \langle e, \ st \rangle \rightarrow_e (v, \ st^{'}}{E_{veaf} \vdash \langle (e), \ st \rangle \rightarrow_e \ (v, \ st^{'})}	
\]

\[
[CALL-FUNC]
\dfrac{E_{eaf} \ \vdash \ \langle active \ B \ end, \ st^{''}, \ E_v^{''} \rangle \ \Rightarrow^{*} \ (st^{'}, E_l) \qquad E_{veaf} \ \vdash \ \langle e_i, \ st_i \rangle \ \rightarrow_e \ (v_i, \ st_{i+1})}{E_{veaf} \ \vdash \ \langle call \ f(x_1 ... x_n), \ st \rangle \ \rightarrow_e \ (v, \ st^{'})}
\]
\begin{math}
	Where
	\quad E_f(f) = (B, (x_1 ... x_n), E_v^{'}) \newline
	\quad st_1 = st \newline
	\quad l_1 = E_v(next), l_{j+1} = new(l_j) for j \in {1 ... n} \newline
	\quad E_v^{''} = E_v^{'}[x_1 \mapsto l_1] ... [x_n \mapsto l_n][next \mapsto l_{n+1}] \newline
	\quad st^{''} = St_{n+1}[l_1 \mapsto v_1] ... [l_n \mapsto v_n] \newline
	\quad v = st^{'}(r)
\end{math}

\subsection{Arithmetic expressions}
\[
[ASM-EXPR] \quad	
\dfrac{E_{veaf} \ \vdash \langle a_1, \ st \rangle \rightarrow_{e} (v_1, st^{''}) \qquad E_{veaf} \ \vdash \langle a_2, \ st^{''}\rangle \rightarrow_e (v_2, \ st^{'})}{E_{veaf} \ \vdash \ \langle a_1 \ op \ a_2, st \rangle \ \rightarrow_{e} (v, st^{'})}	
\]
\begin{math}		
op \ \in \ \{ \ +, \ -, \ * \ \} \newline		
v \ = \ v_1 \ op \ v_2		
\end{math}

\[
[DM-EXPR] \quad
\dfrac{E_{veaf} \ \vdash \ \langle a_1, \ st \rangle \ \rightarrow_{e} (v_1, st^{''}) \qquad E_{veaf} \ \vdash \ \langle a_2, \ st^{''} \rangle \ \rightarrow_e (v_2, \ st^{'})}{E_{veaf} \ \vdash \ \langle a_1 \ op \ a_2, st \rangle \ \rightarrow_{e} (v, st^{'})}	
\]
\begin{math}		
op \ \in \ \{ \ /, \ \% \ \} \ | v \not= \ 0 \newline 		
v \ = \ v_1 \ op \ v_2		
\end{math} 

\subsection{Boolean expressions}
Most of the boolean expressions look alike even with the value of true/false, except the AND/OR expressions. The constructs of AND/OR uses short-circuiting, since in the OR expression if the first value is evaluated to true, there is no need to evaluate the second part of the expression.
\[
[EXPR^{IS=}_{\ \top}] \quad	
\dfrac{E_{veaf} \ \vdash \ \langle e_1, \ st \rangle \ \rightarrow_e (v_1, st^{''}) \qquad E_{veaf} \ \vdash \ \langle e_2, \ st^{''} \rangle \ \rightarrow_e (v_2, \ st^{'})}{E_{veaf} \ \ \vdash \ \ \langle e_1 \ IS= e_2, \ st \rangle \ \ \rightarrow_e (\top, \ st^{'})}	
\]
\begin{math}		
v_1 \ = \ v_2 		
\end{math}	
\[
[EXPR^{IS=}_{\ \bot}] \quad	
\dfrac{E_{veaf} \ \vdash \ \langle e_1, \ st \ \rangle \rightarrow_e (v_1, st^{''}) \qquad E_{veaf} \ \vdash \ \langle e_2, \ st^{''} \rangle \ \rightarrow_e (v_2, \ st^{'})}{E_{veaf} \ \ \vdash \ \langle e_1 \ IS= e_2, \ st \rangle \ \rightarrow_e (\bot, \ st^{'})}	
\]
\begin{math}		
v_1 \ \not= \ v_2		
\end{math}
\[	
[EXPR^{NOT=}_{\ \top}] \quad	
\dfrac{E_{veaf} \ \vdash \ \langle e_1, \ st \rangle \ \rightarrow_e (v_1, st^{''}) \qquad E_{veaf} \ \vdash \ \langle e_2, \ st^{''} \rangle \ \rightarrow_e (v_2, \ st^{'})}{E_{veaf} \ \ \vdash \ \langle e_1 \ NOT= e_2, \ st \rangle \ \rightarrow_e (\top, \ st^{'})}	
\]	
\begin{math}		
v_1 \ \not= \ v_2		
\end{math} 
\[
[EXPR^{NOT=}_{\ \bot}] \quad	
\dfrac{E_{veaf} \ \vdash \ \langle e_1, \ st \rangle \ \rightarrow_e (v_1, st^{''}) \qquad E_{veaf} \ \vdash \ \langle e_2, \ st^{''} \rangle \ \rightarrow_e (v_2, \ st^{'})}{E_{veaf} \ \ \vdash \ \langle e_1 \ NOT= e_2, \ st \rangle \ \rightarrow_e (\bot, \ st^{'})}	
\]	
\begin{math}		
v_1 \ = \ v_2		
\end{math} 
\[	
[EXPR^{\ >}_{\ \top}] \quad	
\dfrac{E_{veaf} \ \vdash <e_1, \ st> \rightarrow_e (v_1, st^{''}) \qquad E_{veaf} \ \vdash <e_2, \ st^{''} \rightarrow_e (v_2, \ st^{'})}{E_{veaf} \ \vdash \ <e_1 \ > e_2, \ st> \ \rightarrow_e (\top, \ st^{'})}	
\]
\begin{math}		
v_1 \ > \ v_2		
\end{math} 
\[	
[EXPR^{\ >}_{\ \bot}] \quad	
\dfrac{E_{veaf} \ \vdash \ \langle e_1, \ st \rangle \ \rightarrow_e (v_1, st^{''}) \qquad E_{veaf} \ \vdash \ \langle e_2, \ st^{''} \rangle \ \rightarrow_e (v_2, \ st^{'})}{E_{veaf} \ \vdash \ \langle e_1 \ > e_2, \ st \rangle \ \rightarrow_e (\bot, \ st^{'})}
\]
\begin{math}		
v_1 \ \not> \ v_2		
\end{math} 
\[	
[EXPR^{\ <}_{\ \top}] \quad	
\dfrac{E_{veaf} \ \vdash \ \langle e_1, \ st \rangle \ \rightarrow_e (v_1, st^{''}) \qquad E_{veaf} \ \vdash \ \langle e_2, \ st^{''} \rangle \ \rightarrow_e (v_2, \ st^{'})}{E_{veaf} \ \vdash \ \langle e_1 \ < e_2, \ st \rangle \ \rightarrow_e (\top, \ st^{'})}
\]
\begin{math}		
v_1 \ < \ v_2		
\end{math} 
\[	
[EXPR^{\ <}_{\ \bot}] \quad	
\dfrac{E_{veaf} \ \vdash \ \langle e_1, \ st \rangle \ \rightarrow_e (v_1, st^{''}) \qquad E_{veaf} \ \vdash \ \langle e_2, \ st^{''} \rangle \ \rightarrow_e (v_2, \ st^{'})}{E_{veaf} \ \ \vdash \ \langle e_1 \ < e_2, \ st \rangle \ \rightarrow_e (\bot, \ st^{'})}	
\]
\begin{math}		
v_1 \ \not< \ v_2		
\end{math}
\[	
[EXPR^{\ >=}_{\ \ \top}] \quad	
\dfrac{E_{veaf} \ \vdash \ \langle e_1, \ st \rangle \ \rightarrow_e (v_1, st^{''}) \qquad E_{veaf} \ \vdash \ \langle e_2, \ st^{''} \rangle \ \rightarrow_e (v_2, \ st^{'})}{E_{veaf} \ \ \vdash \ \langle e_1 \ >= e_2, \ st \rangle \ \rightarrow_e (\top, \ st^{'})}	
\]
\begin{math}		
v_1 \ \ge \ v_2		
\end{math} 
\[	
[EXPR^{\ >=}_{\ \ \bot}] \quad	
\dfrac{E_{veaf} \ \vdash \ \langle e_1, \ st \rangle \ \rightarrow_e (v_1, st^{''}) \qquad E_{veaf} \ \vdash \ \langle e_2, \ st^{''} \rangle \ \rightarrow_e (v_2, \ st^{'})}{E_{veaf} \ \ \vdash \ \langle e_1 \ >= e_2, \ st \rangle \ \rightarrow_e (\bot, \ st^{'})}	
\]
\begin{math}		
v_1 \ \ngeq \ v_2		
\end{math}
\[
[EXPR^{\ <=}_{\ \ \top}] \quad
\dfrac{E_{veaf} \ \vdash \ \langle e_1, \ st \rangle \ \rightarrow_e (v_1, st^{''}) \qquad E_{veaf} \ \vdash \ \langle e_2, \ st^{''} \rangle \ \rightarrow_e (v_2, \ st^{'})}{E_{veaf} \ \vdash \ \langle e_1 \ <= e_2, \ st \rangle \ \rightarrow_e (\top, \ st^{'})}
\]
\begin{math}		
v_1 \ \le \ v_2		
\end{math} 
\[	
[EXPR^{\ <=}_{\ \ \bot}] \quad	
\dfrac{E_{veaf} \ \vdash \ \langle e_1, \ st \rangle \ \rightarrow_e (v_1, st^{''}) \qquad E_{veaf} \ \vdash \ \langle e_2, \ st^{''} \rangle \ \rightarrow_e (v_2, \ st^{'})}{E_{veaf} \ \vdash \ \langle e_1 \ <= e_2, \ st \rangle \ \rightarrow_e (\bot, \ st^{'})}	
\]
\begin{math}		
v_1 \ \nleq \ v_2		
\end{math}
\[	
[EXPR^{\ \ !}_{\ \ \top}] \quad	
\dfrac{E_{veaf} \ \vdash \ \langle e, \ st \rangle \ \rightarrow_e (\top, st^{'})}{E_{veaf} \ \vdash \ \langle !e, \ st \rangle \ \rightarrow_e (\top, \ st^{'})}	
\] \newline
\[	
[EXPR^{\ \ !}_{\ \ \bot}] \quad	
\dfrac{E_{veaf} \ \vdash \ \langle e, \ st \rangle \ \rightarrow_e (\bot, st^{'})}{E_{veaf} \ \vdash \ \langle !e, \ st \rangle \ \rightarrow_e (\bot, \ st^{'})}	
\]\newline
\[	
[EXPR^{\ AND}_{\ \ \top}] \quad	
\dfrac{E_{veaf} \ \vdash \ \langle e_1, \ st \rangle \ \rightarrow_e \ (\top, \ st^{''}) \qquad E_{veaf} \ \vdash \ \langle e_2, \ st^{''} \ \rightarrow_e \ (v, st^{'})}{E_{veaf} \ \vdash \ <e_1 \ AND \ e_2, \ st> \ \rightarrow_e (v, \ st^{'})}	
\] \newline
\[	
[EXPR^{\ AND}_{\ \ \bot}] \quad	
\dfrac{E_{veaf} \ \vdash <e_1, \ st> \rightarrow_e (\bot, \ st^{'}) }{E_{veaf} \ \vdash \ <e_1 \ AND \ e_2, \ st> \ \rightarrow_e (\bot, \ st^{'})}	
\]\newline
\[	
[EXPR^{\ OR}_{\ \ \top}] \quad	
\dfrac{E_{veaf} \ \vdash <e_1, \ st> \rightarrow_e (\top, \ st^{'}) }{E_{veaf} \ \vdash \ <e_1 \ OR \ e_2, \ st> \ \rightarrow_e (\top, \ st^{'})}	
\]\newline
\[	
[EXPR^{\ OR}_{\ \ \bot}] \quad	
\dfrac{E_{veaf} \ \vdash \ \langle e_1, \ st \rangle \ \rightarrow_e \ (\bot, \ st^{''}) \qquad E_{veaf} \ \vdash \ \langle e_2, \ st^{''} \ \rightarrow_e \ (v, st^{'})}{E_{veaf} \ \vdash \ <e_1 \ OR \ e_2, \ st> \ \rightarrow_e (v, \ st^{'})}
\]